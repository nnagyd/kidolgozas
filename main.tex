\documentclass[12pt]{article}
\usepackage[english]{babel}
\usepackage{t1enc}
\usepackage{geometry}
\usepackage{graphicx}
\usepackage{float}
\usepackage{caption}
\usepackage{amsmath,amsthm,amsfonts,amssymb,amscd}
\usepackage{hyperref}
\usepackage{fancyhdr}
\usepackage{nicefrac}
\usepackage{titlesec}
\usepackage{pdfpages}
\usepackage{mathtools}
\usepackage{siunitx}
\usepackage{pdflscape}
\usepackage{csquotes}


\geometry{
a4paper,
total={165mm,235mm},
left=22mm,
top=33mm,
}

\pagestyle{empty}
\date{}

\pagestyle{fancyplain}
\headheight 35pt
\rhead{\Course}
\chead{\textbf{\Large \Title}}
\lhead{\Date }
\lfoot{}
\cfoot{}
\rfoot{\small\thepage}
\headsep 1.5em


\renewcommand{\d}[1]{\mathrm{d}#1}

\sisetup{exponent-product = \cdot,
         output-product = \cdot,
         per-mode=fraction}


\usepackage[backend=biber,style=numeric,sorting=none]{biblatex}

\addbibresource{parts/literature.bib}

\newcommand{\specialcell}[2][c]{\begin{tabular}[#1]{c}#2\end{tabular}}


%Define those, after that everything is nice and automatic
\newcommand\Title{Group Assignment}
\newcommand\Date{\today}
\newcommand\Name{Ádám Kohajda \\ Dániel Nagy \\ József Szenka \\ László Kocsis \\ Zoltán Hafner}
\newcommand\Course{Marketing}
\newcommand\Neptun{BMEGT20MW01}
\newcommand\CourseNeptun{BMEGT20MW01}

\begin{document}


\frenchspacing
\thispagestyle{empty} %Ne legyen header

\begin{center}
\Large{\Course} \\
\vskip 0.25cm
\large{\CourseNeptun}
\vskip 2cm

\huge{\Title}

\vskip 2cm
\Huge{\textbf{German beers to Hungarian market (webshop)}}

\vskip 3cm
\Large{\Name}\\

\mbox{}
\vfill

\begin{figure}[htb]
   \centering
   \includegraphics[width=0.7\linewidth]{pics/bme-logo.jpg}
\end{figure}


\large{Budapest University of Technology and Economics} \\
\large{Budapest, 2022}


\pagebreak
\setcounter{page}{1}

\end{center}
 %sometimes include only a pdf instead
%\includepdf[pages=-]{parts/task.pdf} %replace this page
\input{parts/02_contentpage.tex}


\section{Introduction}
What is the first thing that comes to Your mind, when talking about Germany? Is it their world-renowned technological achievements, or their spicy cuisine? Whatever it may be, there is no single soul, that has not heard about their famous beer culture.

Germans are one of the largest beer consumers in Europe. According to studies an average German consumes around 140 litres of beer annually. However, they do not only like to drink beer, but also brew it. It is estimated that there are currently around 1300 operating breweries in Germany, that produce more than 110 hectolitre beer every year \cite{statista1}. In order to maintain German beer quality, there is a German Purity Law, which controls the allowed ingredients of a beer. According to the statue may only use malt, hops, yeast and water to brew beer.

With all this in mind, we decided to bring a “sip” of the German beer culture to Hungary. Our mission is to make the century old beer selections available to the Hungarian market. Currently there are only a handful of accessible German beers to the Hungarian consumers. That’s why we would like to broaden the German beer selection. Our solution includes a webshop, through which consumers can order dozens of different German beers.

\section{Analysis}
\subsection{Market segmentation}

Which one is relevant? How to use them? Evaluate them: segment size and growth?
\begin{itemize}
   \item Geographic segmentation is relevant for our marketing plan. People living in rural areas and smaller towns are  less likely to buy our products as they tend to have less disposable income. This segment also uses the internet less and prefers offline shopping. On the other hand inhabitants of larger cities tend to be richer and more active online. The size of this geographic segment according to the Hungarian Central Statistical Office consists of 3.6 million people \cite{ksh1} and it is also growing in Hungary.
   \item Demographic segmentation
   \item Psychographic segmentation
   \item Behavioral segmentation
\end{itemize}

\subsection{Market targeting}

\begin{itemize}
   \item Which segment to target?
   \item How to target them (differentiated, concentrated, undifferentiated)
\end{itemize}

\subsection{Market positioning}
The position statement:
\textit{To quality beer consumers who enjoy a wide variety of German beers, our webshop is the ultimate place that delivers the perfect beer for your every day consumption as well as for special occasions.}


\begin{itemize}
   \item Identifying a set of possible competitive advantages to build a position
   \item Choosing the right competitive advantages
   \item Selecting an overall positioning strategy
   \item Communicating and delivering the chosen position to the market
   \item Positioning statement:  To (target segment and need) our (brand) is (concept) that (point of difference)
\end{itemize}

\begin{itemize}
   \item Who are the competitors
   \item What is the competitive advantage
\end{itemize}

Our main competitors are listed below. The common factor of our competitors is that they do not specialize in German beers.
\begin{itemize}
   \item \texttt{soronline.hu} \cite{soronline}
   \item \texttt{beerselection.hu} \cite{beerselection}
   \item \texttt{beergourmet.hu} \cite{beergourmet}
   \item \texttt{csakajosor.hu} \cite{csakajosor}
\end{itemize}


\section{Summary}
Our main finding is that we should focus our marketing strategy on middle aged men living in larger Hungarian cities. This group of people has the money to buy better quality.


\input{parts/03_bib.tex} %comment it out if not needed

\section{Appendix}


\end{document}
